\section{Notational Conventions}
\hrule
bnet=B net=Bayesian Network
\hrule

Random Variables will be indicated by underlined letters and their values by non-underlined letters. Each node of a bnet will be labelled by a random variable. Thus, $\rvx=x$ means that node $\rvx$ is in state $x$.
\smallskip
\hrule
 $P_\rvx(x)=P(\rvx=x)=P(x)$ is the probability that random variable $\rvx$ equals $x\in S_\rvx$. $S_\rvx$ is the set of states (i.e., values) that $\rvx$ can assume and $n_\rvx = |S_\rvx|$ is the size (aka cardinality) of that set. Hence, 
\beq
\sum_{x\in S_\rvx}P_\rvx(x)=1
\eeq
\hrule
\beq
P_{\rvx,\rvy}(x,y)=P(\rvx=x, \rvy=y)=P(x,y)
\eeq
\beq
P_{\rvx|\rvy}(x|y)=P(\rvx=x| \rvy=y)=P(x|y)=\frac{P(x,y)}{P(y)}
\eeq
\hrule
Kronecker delta function: For $x,y$ in discrete set $S$, 
\beq
\delta(x,y)=\left\{
\begin{array}{l}
1\;{\rm if}\; x=y
\\
0 \;{\rm if}\; x\neq y
\end{array}
\right.
\eeq
\hrule
Dirac delta function: For $x,y\in\RR$,
\beq
\int^{+\infty}_{-\infty}dx\;\delta(x-y)f(x)=f(y)
\eeq
\hrule
Indicator function:
\beq
\indi(\cals)=\left\{
\begin{array}{l}
1\;{\rm if\; \cals\; is\; true} 
\\
0 \;{\rm if \;\cals\; is \;false}
\end{array}
\right.
\eeq
For example, $\delta(x,y)=\indi(x=y)$.
\hrule
\beq
\vec{x}= (x[0], x[1], x[2] \ldots, x[nsam(\vecx)-1])=x[:]
\eeq

 $nsam(\vecx)$ is the number of samples  of $\vecx$. $\rvx[i]$ are i.d.d. (independent identically distributed) samples with

 \beq
x[i]\sim P_\rvx\;\;({\rm i.e.}\; P_{\ul{x[i]}}=P_\rvx)
\eeq

\beq
P(\rvx=x)=\frac{1}{nsam(\vecx)}\sum_i \indi(x[i]=x)
\eeq 

If we use two sampled variables, say $\vecx$ and $\vecy$, in a given bnet, their number of samples $nsam(\vecx)$ and $nsam(\vecy)$ need not be equal.
\hrule
\beq
P(\vecx) = \prod_i P(x[i])
\eeq

\beq
\sum_\vecx = \prod_i\sum_{x[i]}
\eeq

\beq
\partial_\vecx = 
[\partial_{x[0]}, \partial_{x[1]},\partial_{x[2]}, \dots, \partial_{x[nsam(\vecx)-1]}]
\eeq
\hrule 
\beqa
P(\vecx)&\approx& [\prod_x P(x)^{P(x)}]^{nsam(\vecx)} \\
&=& e^{nsam(\vecx)\sum_x P(x)\log P(x)}\\
&=& e^{-nsam(\vecx)H(P_\rvx)}
\eeqa

\hrule 

\beq
f^{[1, \partial_x, \partial_y]}(x,y) =
[f, \partial_x f , \partial_y f]
\eeq

\beq
f^+=
f^{[1, \partial_x, \partial_y]}
\eeq
\hrule
For probabilty distributions $p(x), q(x)$ of $x\in S_\rvx$
\begin{itemize}
\item 
Entropy:
\beq
H(p)=-\sum_x p(x)\log p(x)\geq 0
\eeq

\item
Kullback-Liebler divergence:

\beq
D_{KL}(p\parallel q)=\sum_{x} p(x)\log \frac{p(x)}{q(x)}\geq 0
\eeq
\item 
Cross entropy:
\beqa
CE(p\rarrow q) &=& -\sum_x p(x)\log q(x)\\
&=& H(p) + D_{KL}(p\parallel q)
\eeqa
\end{itemize}

\hrule
Normal Distribution: $x, \mu, \sigma\in \RR$, $\sigma >0$

\beq 
\caln(\mu, \sigma)(x)=
\frac{1}{\sigma\sqrt{2\pi}}
e^{-\frac{1}{2}\left(
\frac{x-\mu}{\sigma}\right)^2}
\eeq
\hrule
Uniform Distribution: $a<b$, $x\in [a,b]$

\beq
\calu(a,b)(x) =
\frac{1}{b-a}
\eeq